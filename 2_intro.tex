\chapter{Introduction}

Understanding Nature from the study of its microscopic constituents is the goal of particle physics. The knowledge about elementary particles obtained through decades of experimental discoveries, accompanied by theoretical and technological developments, allowed to establish the standard model \cite{Novaes:1999yn} as the main theory of particle physics. The model embeds two quantum field theories: the quantum chromodynamics \cite{Muta:1998vi} for strong interactions and the Glashow--Weinberg--Salam theory \cite{Bilenky:1982ms} of electroweak interactions.

The quantum electrodynamics \cite{qcdFeynman} that describes the electromagnetic interaction with astonishing precision was considered a prototype gauge theory for the other interactions. After the discovery of the asymptotic freedom of the non-abelian gauge theories, a reliable theory for the strong interactions at short distance became possible. Based on the $SU(3)_C$ color group, the quantum chromodynamics (QCD) was very successful describing the strong phenomena at high energies. On the other hand, the weak force has a significant influence only at distances of hundredths of the radius of a proton, and its short range implies that the virtual particles exchanged in weak interactions must be very massive. In fact, the weak bosons are around ninety times heavier than a proton. 

In 1967 Abdus Salam and Steven Weinberg proposed a model of the weak interaction in which the gauge bosons acquire mass through the Higgs mechanism \cite{Higgs:1966ev, Englert:1964et}. They explored the same gauge group proposed in 1961 by Sheldon Glashow \cite{Glashow:1961tr}, the $SU(2)_L\times U(1)_Y$. In order to preserve local symmetries, four gauge bosons are introduced, at the outset, all of them massless. The spontaneous symmetry breaking is implemented by a potential constructed out of a Higgs complex scalar doublet, represented by four scalar particles. Three of the Higgs fields are absorbed by the gauge bosons $W^+$, $W^-$, and $Z$. The fourth gauge boson remains massless, being associated with the photon, the mediator of the electromagnetic interaction. The three scalar particles that lend mass to the gauge bosons disappear from the physical spectrum, but one neutral scalar, the Higgs particle, remains in the physical spectrum and should be observed in experiments.  
 
The first decisive test of the electroweak model was the experimental confirmation of the existence of weak interactions without exchange of electrical charge. Events mediated by neutral currents represented by the exchange of the $Z$ boson, were observed in 1973 at CERN's Gargamelle neutrino experiment \cite{Hasert:1973ff}. Without the $Z$ contribution, any weak interaction would necessarily entail an exchange of electric charge by the $W^{\pm}$ bosons. However, particles interacting through neutral weak currents keep their original identities, therefore, the $Z$ interaction is flavor diagonal. 

In the next decade, the four collaborations of the large Electron--Positron Collider (LEP) were able to deeply scrutinize several aspects of the standard model. They measured almost 20 parameters related to the standard model and arrived to an impressive agreement between experimental data and theoretical predictions. 

Another key prediction of the standard model is the Higgs boson, whose mass is not predicted by the theory. After 45 years of intensive search for this particle, in 2012, both the CMS \cite{Chatrchyan:2012xdj} and ATLAS \cite{Aad:2012tfa} collaborations of the Large Hadron Collider (LHC) presented very strong evidences of the Higgs boson discovery. A boson with mass around 125 GeV sharing much of the characteristics of the Higgs showed up in the $\gamma\gamma$ and four leptons final states.
 
Despite its great success describing all the experimental data at present, the standard model has some intrinsic problems. For instance, it does not predict the value of the masses of the elementary particles and does not furnish an explanation for the huge amplitude of the mass spectrum. The model contains a large number of parameters, does not explain the generation structure, and has difficulties associated with hierarchy and fine-tuning problems. It does not incorporate gravity and fails to unify the electroweak and strong forces. The standard model does not have any candidate that could explain the existence of the dark matter either.

There are alternative theories that could solve some of the standard model problems. Much of the current theoretical efforts involve new physics at the TeV scale, such as supersymmetry \cite{Martin:1997ns}, compact extra dimensions \cite{Csaki:2004ay}, or composite Higgs models \cite{Agashe:2004rs} for spontaneous symmetry breaking. 

Another example of beyond the standard model (BSM) theory is the Randall-Sundrum (RS) warped extra-dimensional model \cite{Randall:1999ee,Randall:1999vf}, which offers an unification approach of all interactions including gravity. An extension of the original RS model, the bulk graviton \cite{Agashe:2007zd}, is particularly important from the experimental point of view because it predicts the existence of heavy resonances coupling to pairs of vector bosons (ZZ, WZ, and WW), generically called diboson (VV) resonances.

The experimental approach to the searches is independent of the details of the specific production or decay patterns, allowing the LHC to explore different hypotheses and set constraints to several BSM theories. The direct observation of TeV resonances is technically possible due to the excellent detection capability and general purpose of the CMS experiment. CMS was conceived to discover not only the Higgs boson, by measuring essential observables to identify leptons, photons, hadrons, and missing transverse energy, in a range of energy surpassing the TeV scale. 

Given a massive boson $V$ decaying through a hadronic channel, $V \rightarrow qq$, if the mother particle $V$ has high enough boost the final state particles may often be reconstructed as a single hadronic jet. Separating jets originating from boson decays from those that come from standard QCD processes requires studying the inner structure of the jet in question. The jet mass can be used to discriminate vector bosons as long as the soft radiation, underlying event, and pileup contributions, are carefully removed from the radius of the jet. The procedure to remove the undesirable contamination is called \emph{pruning} and consists on rerunning the jet algorithm and vetoing on these components, resulting in a pruned jet \cite{Ellis:2009su}.

%Given a high mass resonance $X$, the final state particles may often be produced with sufficient boost to appear as a single jet, as for the case of hadronic decays. Identifying those jets requires effective techniques to separate vector boson decays from QCD background. The jet mass can be use to discriminate vector bosons as long as the soft radiation, underlying event, and pileup contributions, be carefully removed from the radius of the jet. The procedure to remove the undesirable contamination operates by rerunning the jet algorithm and vetoing on these components, resulting in a pruned jet \cite{Ellis:2009su}.

The extended capabilities of the CMS experiment allow us to perform BSM searches. In particular, we explore the production of a unknown resonance $X$ that decays to a pair of vector bosons, one being a Z boson, with two leptons and one jet in the final state, $X \rightarrow ZV \rightarrow \ell \ell + $ jet. The leptons can be electrons or muons, and their signature corresponds to the decay of a Z boson, $Z\rightarrow \ell \ell$. On the other hand, there is a pruned jet associated to the hadronic decay of the vector boson V (W or Z), characterized by a two-prong jet substructure produced by the initial quarks, $V \rightarrow qq \rightarrow$ jet.

Using pruned jets and good identified leptons in the final state, we construct the diboson invariant mass $\mZV$ to serve as experimental signature of the unknown resonance mass $M_X$. The analysis targets the search for bumps or excesses in the distribution of the diboson invariant mass, considering standard model processes Z+jets, VV, and $t\bar{t}$, as background, and a bulk graviton as signal, $X = G_{\rm bulk}$. The adoption of other signal models involving charged resonances ($X=W'$) is still possible given the general treatment of the hadronic V boson reconstruction; however, our results are specialized to the neutral resonance scenario.
 %with mass in the range between 800 GeV and 2.5 TeV, varying in steps of 100 GeV, as signal.

In Chapter 2, we present a synthetic review of the standard model, quantum chromodynamics, the electroweak theory, physics beyond the standard model, and the bulk graviton model. Chapter 3 gives an overview of the LHC and outlines the CMS experiment. Chapter 4 contains the definition of the ZZ semi-leptonic channel and results from previous searches. Control distributions using real data are presented in Chapter 5, followed by a detailed discussion about background estimation in Chapter 6. Results including the computation of confidence limits are presented in Chapter 7, and finally, the conclusions are given in Chapter 8.  

