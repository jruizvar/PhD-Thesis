\def\baselinestretch{1.2}
%\setlength{\textheight}{25 true cm}
%\setlength{\textwidth}{15. true cm}
\hoffset=-1.0 true cm
\voffset=-2 true cm
%\textheight=21cm
\topmargin=1.0cm
\thispagestyle{empty}
\def\thefootnote{\fnsymbol{footnote}}

%\begin{titlepage}
\thicklines
\begin{picture}(370,60)(0,0)
\setlength{\unitlength}{1pt}
%
\put(40,53){\line(2,3){15}}
\put(40,53){\line(5,6){19}}
\put(40,53){\line(1,1){27}}
\put(40,53){\line(6,5){33}}
\put(40,53){\line(3,2){25}}
\put(40,53){\line(2,1){19}}
%
\put(40,53){\line(5,-6){17}}
\put(40,53){\line(1,-1){22}}
\put(40,53){\line(6,-5){30}}
\put(40,53){\line(3,-2){22}}
%
\put(40,53){\line(-2,1){15}}
\put(40,53){\line(-3,1){23}}
\put(40,53){\line(-4,1){26}}
\put(40,53){\line(-6,1){36}}
\put(40,53){\line(-1,0){40}}
\put(40,53){\line(-6,-1){32}}
\put(40,53){\line(-3,-1){20}}
\put(40,53){\line(-2,-1){10}}
%
\put(75,45){\Huge \textsf{\textbf{ IFT}}}
\put(180,56){\small \sf Instituto de F\'\i sica Te\'orica}
\put(165,42){\small \sf Universidade Estadual Paulista} 
%\put(0,0){\line(1,0){375}}
\put(-25,2){\line(1,0){433}}
\put(-25,-2){\line(1,0){433}}
\end{picture}  

%%%%%%%%%%%%%%%%%%%%%%%%%%%%%%%%
%% Escolha a opcao apropriada %%
%%%%%%%%%%%%%%%%%%%%%%%%%%%%%%%%

\vskip .3cm
%\noindent
%{MASTER DISSERTATION}
%%%%%%%%%%% Numero da Dissertacao %%%%%%%%%%%%%
%\hfill    IFT--M. 009/13\\

\noindent
{\sf DOCTORAL THESIS}
%%%%%%%%%%%%%%% Numero da Tese %%%%%%%%%%%%% 
\hfill   {\sf  IFT--T.000/00 ~}\\

\vspace{3cm}
\begin{center}
%%%%%%%% TITULO DA DISSERTACAO OU TESE %%%%%%
{\LARGE \textsf{\textbf{Search for New Resonances in the \\[0.5cm] Merged Jet plus Dilepton Final State in CMS}}}
%{\large \bf Search for New Resonances in the \\ Merged Jet + Dilepton Final State in CMS}
%%%%%%%%%%%%%%%%%%%%%%%%%%%%%%%%%%

\vspace{1.5cm}
%%%%%%%%%%%%%%% NOME DO AUTOR %%%%%%%%
{\Large \textsf{\textit{Jos\'e Cupertino Ruiz Vargas}}}
%%%%%%%%%%%%%%%%%%%%%%%%%%%%%%%%%
\end{center}

\vskip 3cm
\hfill \textsf{\textbf{Advisor}}
\vskip 0.1cm
%%%%%%%%%%%%% NOME DO ORIENTADOR %%%%%%%%%
\hfill {\sf Prof. S\'ergio F. Novaes}
\vskip 1cm
\hfill \textsf{\textbf{Co-advisor}}
\vskip 0.1cm
\hfill {\sf Dr. Thiago R. F. P. Tomei}
%%%%%%%%%%%%%%%%%%%%%%%%%%%%%%%%%
\vfill
\begin{center}
%%%%%%%%%%%%%%%%%% DATA %%%%%%%%%%%%
{\sf May 2017}
%%%%%%%%%%%%%%%%%%%%%%%%%%%%%%%%%
\end{center}

\newpage

\thispagestyle{empty}

%\begin{titlepage}
\thicklines
\begin{picture}(370,60)(0,0)
\setlength{\unitlength}{1pt}
%
\put(40,53){\line(2,3){15}}
\put(40,53){\line(5,6){19}}
\put(40,53){\line(1,1){27}}
\put(40,53){\line(6,5){33}}
\put(40,53){\line(3,2){25}}
\put(40,53){\line(2,1){19}}
%
\put(40,53){\line(5,-6){17}}
\put(40,53){\line(1,-1){22}}
\put(40,53){\line(6,-5){30}}
\put(40,53){\line(3,-2){22}}
%
\put(40,53){\line(-2,1){15}}
\put(40,53){\line(-3,1){23}}
\put(40,53){\line(-4,1){26}}
\put(40,53){\line(-6,1){36}}
\put(40,53){\line(-1,0){40}}
\put(40,53){\line(-6,-1){32}}
\put(40,53){\line(-3,-1){20}}
\put(40,53){\line(-2,-1){10}}
%
\put(75,45){\Huge \textsf{\textbf{ IFT}}}
\put(180,56){\small \sf Instituto de F\'\i sica Te\'orica}
\put(165,42){\small \sf Universidade Estadual Paulista} 
%\put(0,0){\line(1,0){375}}
\put(-25,2){\line(1,0){433}}
\put(-25,-2){\line(1,0){433}}
\end{picture}  

%%%%%%%%%%%%%%%%%%%%%%%%%%%%%%%%
%% Escolha a opcao apropriada %%
%%%%%%%%%%%%%%%%%%%%%%%%%%%%%%%%

\vskip .3cm
%\noindent
%{DISSERTA\c C\~AO DE MESTRADO}
%%%%%%%%%%% Numero da Dissertacao %%%%%%%%%%%%%
%\hfill    IFT--M. 009/13\\

\noindent
{\sf TESE DE DOUTORAMENTO}
%%%%%%%%%%%%%%% Numero da Tese %%%%%%%%%%%%% 
\hfill    {\sf IFT--T.000/00 ~}\\

\vspace{3cm}
\begin{center}
%%%%%%%% TITULO DA DISSERTACAO OU TESE %%%%%%
{\Large \textsf{\textbf{Busca de Novas Resson\^ancias Decaindo em \\[0,5cm] Jato Hadr\^onico e Dois Leptons no Experimento CMS}}}
%{\large \bf Busca de Novas Resson\^ancias no Estado Final com um Jato Hadr\^onico e Dois leptons no Experimento CMS}
%%%%%%%%%%%%%%%%%%%%%%%%%%%%%%%%%%

\vspace{1.5cm}
%%%%%%%%%%%%%%% NOME DO AUTOR %%%%%%%%
{\Large \textsf{\textit{Jos\'e Cupertino Ruiz Vargas}}}
%%%%%%%%%%%%%%%%%%%%%%%%%%%%%%%%%
\end{center}

\vskip 3cm
\hfill \textsf{\textbf{Orientador}}
\vskip 0.1cm
%%%%%%%%%%%%% NOME DO ORIENTADOR %%%%%%%%%
\hfill {\sf Prof. S\'ergio F. Novaes}
\vskip 1cm
\hfill \textsf{\textbf{Co-Orientador}}
\vskip 0.1cm
\hfill {\sf Dr. Thiago R. F. P. Tomei}
%%%%%%%%%%%%%%%%%%%%%%%%%%%%%%%%%
\vfill
\begin{center}
%%%%%%%%%%%%%%%%%% DATA %%%%%%%%%%%%
{\sf Maio de 2017}
%%%%%%%%%%%%%%%%%%%%%%%%%%%%%%%%%
\end{center}

\newpage

\pagenumbering{roman}

\begin{center}
{\LARGE \textsf{\textbf{Acknowledgements}}}
\end{center}
\vskip 2.0cm
%%%%%%%%%%%% AGRADECIMENTOS %%%%%%%%%%%%%

I would like to thank my supervisor Prof. S\'ergio Novaes, for his dedication, patience, and for introducing me to the CMS Collaboration.

I would like to thank my co-advisor, Thiago Tomei, for the guidance, expertise, and bright ideas.

To my family: father, mother and brothers, for their encouragement and especially because they always trusted me.

I would like to thank all SPRACE team: Angelo, Cesar, Chang-Seong, David, Eduardo, Pedro, Sandra, Sudha, and Sunil, for their continuous support.

Thanks to my friends at IFT: Carlisson, Jos\'{e} David, Fernando, and Prieslei, for their camaraderie.

Finally, I would like to thank FAPESP funding agency for the financial support in Brazil, and the European Particle Physics Latin American Network (EPLANET) for covering my living expenses at CERN during my one-year internship.

%%%%%%%%%%%%%%%%%%%%%%%%%%%%%%%%%%%%%

\newpage

\begin{center}
{\LARGE \textsf{\textbf{Resumo}}}
\end{center}
\vskip 2.0cm

%%%%%%%%%%%%%%%%%% RESUMO %%%%%%%%%
Na Organiza\c{c}\~ao Europeia para a Pesquisa Nuclear (CERN), o Large Hadron Collider (LHC) colide grupos de pr\'otons 40 milh\~oes de vezes por segundo a uma energia de 13 TeV. Operando junto ao LHC, o Compact Muon Solenoid (CMS) \'e um detector projetado para detectar uma ampla gama de part\'iculas produzidas nessas colis\~oes.  As part\'iculas produzidas em cada colis\~ao s\~ao observadas nos subdetectores na busca de pistas sobre a Natureza no seu n\'ivel mais fundamental.

Apesar do modelo padr\~ao das part\'iculas elementares ter sido testado em uma variedade de experimentos de altas energias, um dos principais objetivos do LHC \'e a busca de uma nova f\'isica, al\'em daquela prevista por essa teoria. 

Nesse trabalho analisamos os dados de colis\~oes pr\'oton-pr\'oton produzidos pelo LHC operando com energia de centro de massa de 13 TeV e coletados pelo CMS em 2015. O presente estudo envolve a busca de uma nova resson\^ancia que n\~ao haveria sido observada previamente, decaindo em um par de b\'osons vetoriais.

Os resultados s\~ao interpretados no contexto do modelo de dimens\~oes extras deformadas de Randall-Sundrum, distinguindo as hip\'oteses de fundo (modelo padr\~ao) e fundo mais sinal (modelo padr\~ao + graviton). Nenhuma evid\^encia da exist\^encia de uma part\'icula com as caracter\'{\i}sticas do graviton de Randall-Sundrum foi encontrada, levando \`a conclus\~ao de que a rea\c{c}\~ao estudada n\~ao contraria as predi\c c\~oes  do modelo padr\~ao.

%%%%%%%%%%%%%%%%%%%%%%%%%%%%%%%%

\vskip 1.0cm
%%%%%%%%%%%%% PALAVRAS CHAVE %%%%%%%%
%%%%%% SEPARADAS POR PONTO-E-VIRGULA %%%%%
\noindent
{\bf Palavras Chaves}: F\'isica de Altas Energias; F\'isica de Part\'iculas; Colisores Hadr\^onicos; F\'isica Al\'em do Modelo Padr\~ao.
\vskip 0.5cm
%%%%%%%%% AREAS DO CONHECIMENTO %%%%%%%
%%%%%% SEPARADAS POR PONTO-E-VIRGULA %%%%%
\noindent
{\bf \'Areas do conhecimento}: F\'isica de Altas Energias.

\newpage

\begin{center}
{\LARGE \textsf{\textbf{Abstract}}}
\end{center}
\vskip 2.0cm

%%%%%%%%%%%%%%%% ABSTRACT %%%%%%%%%%

At the European Organization for Nuclear Research (CERN), the Large Hadron Collider (LHC) smashes groups of protons 40 million times per second at an energy of 13 TeV. Operating at the LHC, the Compact Muon Solenoide (CMS) is a multipurpose detector conceived to identify a large variety of particles produced in such collisions. The produced particles are observed at the sub-detectors in search of clues about Nature at the most fundamental level.

In spite of the impressive agreement of the standard model with all the experimental results obtained so far, one of the main aim of the LHC is the search of new physics beyond the one foreseen by this theoretical model.

In this work, we analyze the result of proton--proton collisions delivered by the LHC operating at centre-of-mass energy of 13 TeV and collected by CMS during 2015. The channel under study involves the search for possible new resonance decaying into a pair of vector bosons. 

The results are interpreted in the context of the Randall-Sundrum warped extra dimensions model, distinguishing between the  hypotheses of background only (standard model) and background plus signal (standard model + graviton).  No evidence of the existence of a graviton-like particle was found, leading to the conclusion that this reaction does not challenge the predictions of the standard model .

%Upper limits at 95\% confidence level are set for the production cross section of $X$ decaying to a pair of Z bosons as function of the resonance mass. The results are interpreted in the context of the Randall-Sundrum warped extra dimensions model in the mass range between 550 -- 2500 GeV. A localised excess at 650 GeV is observed in the data with global significance of 3.5 $\sigma$. 

\vskip 1.0cm
%%%%%%%%%%%%% PALAVRAS CHAVE %%%%%%%%
%%%%%% SEPARADAS POR PONTO-E-VIRGULA %%%%%
\noindent
{\bf Key Words}: High Energy Physics; Particle Physics; Hadron Colliders, Physics Beyond the Standard Model.
\vskip 0.5cm
%%%%%%%%% AREAS DO CONHECIMENTO %%%%%%%
%%%%%% SEPARADAS POR PONTO-E-VIRGULA %%%%%
\noindent
{\bf Research area}: High Energy Physics.

%%%%%%%%%%%%%%%%%%%%%%%%%%%%%%%%

\vfill \eject
